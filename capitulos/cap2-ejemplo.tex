\cleardoublepage
\thispagestyle{empty}
\hbox{ }
\cleardoublepage

\chapter{NOMBRE DEL CAPITULO 2}\label{cap2}

\begin{flushright}
INSERTAR CITA RELEVANTE (OPCIONAL)
\end{flushright}
 
\section{SECCION 1: ALGUNAS ECUACIONES}

\subsection{Serie geom\'etrica}

Sea $r\in (0,1)$ se tiene que la suma de exponentes consecutivos de $r^i$ converge, es decir
\begin{equation}
\sum_{i=0}^{\infty} r^i = \frac{1}{1-r}
\end{equation}
notamos que $r$ puede crecer de manera tal que
$$
\lim_{r\to \infty} \frac{1}{1-r} = 0
$$
o escrito de otra manera
$$
\sum_{i=0}^{\infty} r^i = \frac{1}{1-r} \stackrel{r\to \infty}{\longrightarrow} 0
$$

\subsection{Distribuci\'on Normal}

\begin{equation}
\mathcal{N}(\mu,\sigma^2) = \int\limits_{-\infty}^{x} \frac1{\sigma\sqrt{2\pi}}\: \exp{-\frac{1}{2}\left(\frac{t-\mu}{\sigma}\right)^2}\: dt
\end{equation}